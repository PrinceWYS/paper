\chapter{绪论}
\section{课题背景及研究意义}

人类生活在三维空间中,并且对三维环境的感知与理解是人类认知的重要组成部分。
随着计算机视觉与人工智能技术的飞速发展,三维场景的高保真重建已经成为连接物理世界与数字世界的关键技术之一。
高质量的三维重建不仅在虚拟现实、增强现实等娱乐领域有着广泛应用,还在建筑设计、文化遗产保护、医疗影像等专业领域展现出巨大的潜力。
这些领域,对三维环境的精确感知与理解提出了日益迫切的需求。
在三维重建领域,传统的基于图像的方法存在诸多局限性,例如对光照变化敏感、难以处理纹理缺失区域等。
近年来,深度学习技术的兴起为三维重建带来了新的机遇。通过深度神经网络,可以实现从单张图像或多视角图像中恢复出高保真三维模型。
然而,如何在保证重建精度的同时提高计算效率,仍是当前研究中的一个关键问题,也是当前学术界和工业界关注的焦点。

对于生活在三维空间中的人类,对于三维环境的理解与描述也是必不可少的一部分。
自然语言作为人类交流的主要方式,其在三维场景理解中的作用日益凸显。
通过自然语言描述三维场景,不仅可以提升人机交互的自然性,还能为智能系统提供更丰富的语义信息,促进其对复杂环境的理解与处理。
因此,探索如何将自然语言与三维场景重建相结合,成为当前研究中的一个重要方向。
开放的自然语言查询,可以帮助用户更方便地获取三维场景中的信息,提升用户体验。
因此,如何在高质量几何表达的基础上构建具备可解释性与交互性的“语义化三维场景”,成为当前的研究热点。

目前,业内的探索集中在利用简单视觉模型进行语义特征提取。
但是,这些仅仅使用简单的视觉模型的方法具有局限性。
首先,这些方法使用的简单视觉模型,在面对复杂的自然语言查询时,往往难以捕捉到丰富的语义信息,导致查询结果的准确性和相关性不足。
并且,大多数方法仅关注于物体级别的语义理解,忽略了场景中更细粒度的语义关系和上下文信息,这限制了系统在复杂场景中的表现能力。
其次,这些方法大多需要训练额外的编码器、解码器或者语义码本,增加了系统的复杂性和计算开销,难以实现高效的实时查询,也限制了跨场景的泛化能力。
此外,一些方法希望通过聚类的方式来提升语义理解的效果。
但它们往往聚焦于三维空间的位置,而忽略了语义信息的多样性和复杂性,导致一些空间距离较为接近但是语义特征大相径庭的部分被错误地聚类。

因此,若能够将更加准确丰富、具有更加层次化的语义信息嵌入到高质量的三维几何表达中,将能够实现更加准确的开放自然语言查询,提升三维场景的理解与交互能力。
这不仅有助于推动三维重建技术的发展,还将为虚拟现实、智能导航等应用领域带来新的突破,满足人们对数字化三维环境的多样化需求。

目前,大语言模型(Large Language Models, LLMs)在自然语言处理领域取得了显著的进展,展示了强大的语言理解和生成能力。
与此同时,多模态大语言模型(Multimodal Large Language Models, MLLMs)通过结合视觉和语言信息,进一步提升了对复杂场景的理解能力。
这些模型在图像描述、视觉问答等任务中表现出色,显示出其在处理多模态数据方面的巨大潜力。MLLMs能够对视觉信息进行深度理解,并输出丰富的自然语言描述,这为三维场景的语义化表示提供了新的思路。
如果能将MLLMs与高质量的三维几何表达相结合,由于自然语言描述的丰富性和灵活性,将有望实现对三维场景的更深入理解和更自然的交互方式。

在本文中,提出了一种将MLLMs与三维几何表达相结合的方法,将MLLMs的强大语义理解能力嵌入到高质量的三维场景表示中。
通过这种结合,将能够实现对三维场景的开放自然语言查询,提升系统对复杂场景的理解和交互能力。
这将为三维重建技术的发展带来新的机遇,推动虚拟现实、智能导航等应用领域的创新发展,具有十分重要的研究意义和应用价值。

\section{国内外研究现状}
\subsection{三维重建}

三维重建旨在从二维的平面图像观测中,恢复出场景的三维几何结构与外观属性,是连接物理世界与数字世界的关键技术之一,在虚拟显示、增强现实、机器人导航及自动驾驶环境感知等领域得到了广泛的应用。
早期的研究主要依赖于多视图几何理论,通过运动恢复结构(Structure from Motion, SfM\cite{schonbergerStructurefromMotionRevisited2016a})与多视图立体视觉(Multi-View Stereo, MVS\cite{furukawaAccurateDenseRobust2010})算法重建显示的点云或者三角网格。
然而,此类方法在处理弱纹理区域、反射表面以及复杂遮挡关系时,往往难以保证重建的完整性与视觉真实感。
近年来,随着深度学习与计算机图形学的深度融合,三维重建技术经历了从传统几何建模到隐式神经表示,再到显式高斯表示的范式转变。

在三维重建的发展历程中,Mildenhall等人于2020年提出的神经辐射场(Neural Radiance Fields, NeRF\cite{mildenhallNeRFRepresentingScenes2020})具有里程碑式的意义。
NeRF摒弃了传统的网格或体素表示,转而使用多层感知机(Multi-Layer Perception, MLP\cite{rumelhartLearningRepresentationsBackpropagating1986})拟合场景的体积密度与颜色函数,结合体渲染技术将场景建模为连续的辐射场函数。
通过体渲染(Volume Rendering)技术,NeRF能够从稀疏的二维图像中合成出具有光照一致性的高保真新视角图像,在视图合成质量上显著优于传统的多视图立体几何方法。
然而,尽管NeRF在渲染质量上取得了重大突破,但是其隐式表达的特性也带来了显著的局限性。
首先,由于需要对每条光线上的大量采样点进行神经网络查询,导致NeRF的训练时间通常以小时计,渲染速度难以达到实时要求,训练与推理效率低。
其次,场景信息被隐式存储在网络权重中,难以像网格或点云那样进行局部的几何修改或物体操作,可编辑性差。

为了解决隐式表示在实时性与可编辑性方面的瓶颈,Kerbl等人于2023年提出了三维高斯溅射(3D Gaussian Splatting, 3DGS\cite{kerbl3DGaussianSplatting2023})。
与NeRF不同,3DGS回归了显式的场景表示,利用一组各向异性的三维高斯椭球(Gaussian Ellipsoids)来参数化场景。
每个高斯点由中心位置、旋转四元数、缩放因子、不透明度以及用于描述视角相关颜色的球谐系数定义。
得益于基于图块(Tile-based)的可微光栅化渲染管线,3DGS避开了NeRF耗时的光线采样步骤,在保持与NeRF相当的高精度的同时,实现了千兆像素级别的实时渲染速度。

更重要的是,3DGS的显式属性为三维场景的语义化提供了天然的物理载体。
相较于NeRF需要在体空间内进行昂贵的语义场积分,3DGS允许将语义特征直接作为高斯点的附加属性进行存储与优化。
通过在可微渲染过程中引入透射率权重驱动的特征嵌入,可以建立起二维图像语义信息与三维高斯点之间的映射关系。
因此,3DGS不仅在几何重建效率上具有显著优势,而且由于其离散与显式的表达特性,成为了构建细粒度、可交互的三维语义场的理想技术路线。

\subsection{MLLMs与开放词汇理解}

近年来,MLLMs的迅猛发展极大地拓宽了计算机视觉任务的语义边界。
以CLIP\cite{radfordLearningTransferableVisual2021}为代表的对比学习框架,通过在海量的图像-文本对上进行预训练,实现了视觉特征与文本嵌入向量的深度对其。
这种跨模态的映射能力使得场景理解不再局限于预定义的有限类别,赋予了系统初步的开放词汇(Open-vocabulary)检索能力。
然而,CLIP及其变体主要关注图像级或者物体级的全局特征,在处理复杂场景中具有精细结构的对象时,其语义表达能力仍显不足。
并且,通过对比学习训练的CLIP无法真正处理长文本,在理解物体的相对位置时表现较差。

随着LLM的兴起,研究者开始通过视觉指令微调将强大的自然语言推理能力引入视觉任务中。
LLaVA\cite{liuVisualInstructionTuning2023}和BLIP-2\cite{liBLIP2BootstrappingLanguageImage2023}等MLLMs通过线性投影层或者Q-Former架构,将视觉编码器的输出映射到LLM的嵌入空间,从而实现了对图像内容的详细描述与交互式的问答。
尽管MLLMs在通用场景理解上取得了突破,但是在细粒度的场景理解中,通用模型往往面临着空间定位与属性理解的瓶颈。

为了解决这一难题,近年来涌现出一批具备区域感知能力的细粒度多模态模型。
其中,Osprey\cite{yuanOspreyPixelUnderstanding2024}通过引入掩码(Mask)引导的特征提取机制,实现了像素级别的细粒度指令微调。
该模型能够针对输入的任意形状掩码,生成高度精确且包含丰富上下文关系的语义描述。
结合分割模型SAM\cite{kirillovSegmentAnything2023}的多粒度分割,可以从图像中提取从宏观物体到微观部件的多层级的语义特征。

\subsection{三维语义场构建}

三维语义场的研究目的在于将开放词汇的理解能力赋予三维的几何表示。在3DGS出现之前,语义场的研究主要聚焦于隐式神经表示。
Semantic-NeRF\cite{zhiInPlaceSceneLabelling2021}通过在NeRF的MLP架构中增加语义分支,实现几何外观与语义的联合建模。
尽管此类工作在多是骄傲语义一致性上表现出色,但受限于体渲染的离散采样机制,在对实时的语义查询与动态的场景编辑方面存在显著的性能瓶颈。

随着3DGS技术的兴起,三维语义场的构建转向了更加高效的显式表达范式。
Gaussian Grouping\cite{yeGaussianGroupingSegment2024}通过引入标识符属性对高斯点进行聚类,实现了初步的物体级分割与编辑。
然而,该方法本质上仍依赖于渲染后的二维模型进行语义推理,并未直线语义特征在三维空间的直接解耦与存储。
随后,LangSplat\cite{qinLangSplat3DLanguage2024}提出将高维CLIP特征直接嵌入到3D高斯点中,利用分级训练和自编码器(Autoencoder)缓解了高维特征带来的显存压力。
尽管LangSplat实现了端到端的语义查询,但是针对每个场景独立训练编码器的模式导致了较高的预处理成本,并且难以有效应对复杂的语义描述。

针对查询效率与空间开销的问题,LEGaussians\cite{shiLanguageEmbedded3D2024}通过语义码本(Codebook)对高维度的语义特征进行量化,而OpenGaussian\cite{wuOpenGaussianPointLevel3D2024}则尝试学习低维映射关系以支持无需渲染的直接查询。
尽管上述工作在效率上有所提升,但是仍普遍存在两大瓶颈:首先,它们的语义大多源于图像级的CLIP特征,缺乏对物体局部细节的深层感知;其次,现有工作在查询阶段大多采用逐点计算模式,在面对百万级的高斯点云时,实时相应速度受到极大的挑战。

\section{主要研究内容与本文工作}

从三维语义场构建当先研究现状来看,基于3DGS的三维语义场重建方法已经可以保证高质量的几何重建,并且对于简单的开放词汇查询也有不错的效果。
目前的主流方法大多采取训练额外的语义码本或者自动编码器将高维的语义信息压缩以减小计算的压力。
然而,在面对复杂的自然语言查询,尤其是涉及到空间位置、物体属性的复杂描述时,这些基于CLIP语义特征的方法往往表现不佳,经常得到不准确或者错误的结果。
例如,对于像“摆放在桌上的红苹果”或者“在筷子旁边的白色纸巾”等描述,这些方法查询到的结果可能发生偏差(如查询结果是场景中的“桌子”或者“筷子”),或者返回查询不到词汇的结果。
并且,目前大多数的研究工作为了降低显存的计算开销,往往要训练额外的语义码本或者编码器。
这增加了预处理的训练开销。
在进行聚类时,这些方法依据的往往是高斯点的空间位置,容易将空间位置相近但是语义信息相差较大的高斯点错误地聚类。

针对上述问题,本文研究并实现了一种融合细粒度多模态大语言模型的三维高斯语义场构建技术。
该技术可以利用MLLM为二维图像生成详细的语义信息,并将语义信息免训练地嵌入到三维高斯点中。
为了满足层次化的语义分类以得到更准确的查询结果,本文提出了一种结合三维高斯点的空间位置与语义信息联合聚类的方法。
在实现了精确的开放词汇查询任务后,该技术可以对场景中的物体进行方便的编辑,包括修改颜色、删除物体等。

% TODO: 技术路线图

本文的主要研究内容和工作简要总结如下:
\begin{enumerate}[label=\arabic*)]
    \item 设计了一种跨视角一致的细粒度特征提取方法。针对三维场景中复杂对象的细粒度理解要求,本文利用分割模型SAM与视频对象分割模型DEVA在多视角图像间建立稳健的掩码关联。在此基础上,引入了细粒度多模态大语言模型Osprey,针对特征区域生成详尽的自然语言描述,并通过预训练的文本编码器将其转化为高维语义特征向量。该方法通过结合像素级掩码与长文本描述,突破了传统方法仅依赖图像级CLIP特征导致的语义理解瓶颈。
    \item 提出了一种空间与语义联合驱动的层次化聚类策略。为了提升大规模场景下的开放词汇查询效率,本文构建了综合空间位置、颜色、法向量以及语义特征的加权邻接图。利用K近邻聚类与图聚类算法,对空间中冗余的高斯点进行多粒度分组,生成层次化的语义场结构。可以根据需求在不同的尺度下进行语义检索。
    \item 设计并实现了针对三维语义场的具体应用——基于精确开放语言查询的场景编辑应用。基于开放语言查询的结果,该应用可以通过对空间中的三维高斯点修改包括颜色、位置、不透明等属性,编辑场景中的物体,如删除物体、修改物体颜色、改变物体位置等。
\end{enumerate}

\section{本文组织架构}

% TODO: 组织架构图
本文全部内容总共分为6章,其组织架构如图\textbf{待修改}所示,各个章节内容具体安排如下:

第1章,绪论。介绍了本文的研究背景和意义,调研并总结了当前三维语义场重建在细粒度表达与构建效率方面的不足,并对本文的研究内容和工作进行概括。

第2章,相关技术综述。介绍了本文中应用到的一些技术,包括3DGS、MLLM以及三维语义场重建的相关技术。
重点介绍了当前基于3DGS的通用图场构建方法在特征嵌入与查询效率、准确率上的局限性,为后续章节提供理论支撑。

第3章,融合多模态大语言模型的多视角一致性语义特征提取。
详细介绍了使用分割模型和视频对象分割模型生成多视角一致性掩码的方法,阐述了使用细粒度MLLM生成精确语义描述并转换为高维语义特征的方法。
最后评估了多模态大语言模型生成语义特征对于复杂词汇查询的准确性,并验证了该方法用于三维语义场重建的有效性。

第4章,空间-语义联合驱动的层次化聚类策略。
详细阐述了一种融合空间位置与高维语义信息特征的图聚类算法,构建了层次化的语义场结构,支持用户在不同粒度下进行交互。
评估了该算法在不同粒度下进行词汇查询的准确性。

第5章,基于开放词汇查询的场景编辑系统的设计与实现。
设计并实现了场景编辑系统,详细阐述了根据开放词汇查询的准确结果通过修改三维高斯点进行场景编辑的方法,并对系统进行了测试,将整体效果及可视化结果进行了展示。

第6章,总结与展望。总结了全文的主要工作和内容,同时指出了目前算法存在的一些问题和可优化的空间,并对未来的工作进行了展望。

\section{本章小结}
本章为绪论部分,首先从三维重建技术与多模态大语言模型的交叉发展触发,阐述了构建具备开放词汇查询能力的三维语义场在智能场景理解、机器人导航等领域的广泛应用前景,并对国内外该领域的研究现状作了阐述。
通过对目前三维语义场重建方法的分析总结,提出了一种新的算法思路,概括了本文接下来所作的工作内容和创新贡献,最后对文章的组织架构进行了编排。

\chapter{相关研究工作}

本章详细介绍了系统相关研究和实现过程中用到的主要技术。首先,本章将对本文设计到的基本的三维高斯喷溅技术进行阐述。其次,针对语义信息的来源,本章将对选用的多模态大语言模型进行介绍。最后,本章将聚焦于三维语义场的构建范式,介绍当前三维语义场的通用构建技术。

\section{三维高斯喷溅}

\section{多模态大语言模型}

\section{三维语义场构建}
\subsection{Gaussian Grouping}

\subsection{Occam's LGS}

\section{本章小结}

\chapter{第三章}

\section{本章小结}

\chapter{第四章}
\section{本章小结}

\chapter{第五章}
\section{本章小结}

\chapter{总结与展望}
\section{本文工作总结}
\section{未来展望}