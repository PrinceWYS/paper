\chapter{绪论}
\section{课题背景及研究意义}

人类生活在三维空间中,并且对三维环境的感知与理解是人类认知的重要组成部分。随着计算机视觉与人工智能技术的飞速发展,三维场景的高保真重建已经成为连接物理世界与数字世界的关键技术之一。高质量的三维重建不仅在虚拟现实、增强现实等娱乐领域有着广泛应用,还在建筑设计、文化遗产保护、医疗影像等专业领域展现出巨大的潜力。这些领域,对三维环境的精确感知与理解提出了日益迫切的需求。
在三维重建领域,传统的基于图像的方法存在诸多局限性,例如对光照变化敏感、难以处理纹理缺失区域等。近年来,深度学习技术的兴起为三维重建带来了新的机遇。通过深度神经网络,可以实现从单张图像或多视角图像中恢复出高保真三维模型。然而,如何在保证重建精度的同时提高计算效率,仍是当前研究中的一个关键问题,也是当前学术界和工业界关注的焦点。

对于生活在三维空间中的人类,对于三维环境的理解与描述也是必不可少的一部分。自然语言作为人类交流的主要方式,其在三维场景理解中的作用日益凸显。通过自然语言描述三维场景,不仅可以提升人机交互的自然性,还能为智能系统提供更丰富的语义信息,促进其对复杂环境的理解与处理。因此,探索如何将自然语言与三维场景重建相结合,成为当前研究中的一个重要方向。开放的自然语言查询,可以帮助用户更方便地获取三维场景中的信息,提升用户体验。因此,如何在高质量几何表达的基础上构建具备可解释性与交互性的“语义化三维场景”,成为当前的研究热点。

目前,业内的探索集中在利用简单视觉模型进行语义特征提取。但是,这些仅仅使用简单的视觉模型的方法具有局限性。首先,这些方法使用的简单视觉模型,在面对复杂的自然语言查询时,往往难以捕捉到丰富的语义信息,导致查询结果的准确性和相关性不足。并且,大多数方法仅关注于物体级别的语义理解,忽略了场景中更细粒度的语义关系和上下文信息,这限制了系统在复杂场景中的表现能力。其次,这些方法大多需要训练额外的编码器、解码器或者语义码本,增加了系统的复杂性和计算开销,难以实现高效的实时查询,也限制了跨场景的泛化能力。此外,一些方法希望通过聚类的方式来提升语义理解的效果。但它们往往聚焦于三维空间的位置,而忽略了语义信息的多样性和复杂性,导致一些空间距离较为接近但是语义特征大相径庭的部分被错误地聚类。

因此,若能够将更加准确丰富、具有更加层次化的语义信息嵌入到高质量的三维几何表达中,将能够实现更加准确的开放自然语言查询,提升三维场景的理解与交互能力。这不仅有助于推动三维重建技术的发展,还将为虚拟现实、智能导航等应用领域带来新的突破,满足人们对数字化三维环境的多样化需求。

目前,大语言模型(Large Language Models, LLMs)在自然语言处理领域取得了显著的进展,展示了强大的语言理解和生成能力。与此同时,多模态大模型(Multimodal Large Models, MLLMs)通过结合视觉和语言信息,进一步提升了对复杂场景的理解能力。这些模型在图像描述、视觉问答等任务中表现出色,显示出其在处理多模态数据方面的巨大潜力。多模态大模型能够对视觉信息进行深度理解,并输出丰富的自然语言描述,这为三维场景的语义化表示提供了新的思路。如果能将多模态大模型与高质量的三维几何表达相结合,由于自然语言描述的丰富性和灵活性,将有望实现对三维场景的更深入理解和更自然的交互方式。

在本文中,提出了一种将多模态大模型与三维几何表达相结合的方法,将多模态大模型的强大语义理解能力嵌入到高质量的三维场景表示中。通过这种结合,将能够实现对三维场景的开放自然语言查询,提升系统对复杂场景的理解和交互能力。这将为三维重建技术的发展带来新的机遇,推动虚拟现实、智能导航等应用领域的创新发展,具有十分重要的研究意义和应用价值。

\section{国内外研究现状}
\subsection{新视角合成与三维重建}

\subsection{多模态大模型与开放词汇理解}
\subsection{三维语义场构建}

我们可以用includegraphics来插入现有的jpg等格式的图片,
如\autoref{fig:zju-logo}所示。

\begin{figure}[htbp]
    \centering
    \includegraphics[width=.3\linewidth]{logo/zju}
    \caption{\label{fig:zju-logo}浙江大学LOGO}
\end{figure}


\subsection{小节标题}

\section{主要研究内容与本文工作}
\section{本文组织架构}
\section{本章小结}

\chapter{相关研究工作}
\section{本章小结}

\chapter{第三章}

\section{本章小结}

\chapter{第四章}
\section{本章小结}

\chapter{第五章}
\section{本章小结}

\chapter{总结与展望}
\section{本文工作总结}
\section{未来展望}